% Please do not change the document class
\documentclass{scrartcl}

% Please do not change these packages
\usepackage[hidelinks]{hyperref}
\usepackage[none]{hyphenat}
\usepackage{setspace}
\doublespace

% You may add additional packages here
\usepackage{amsmath}
\usepackage{graphicx} 
\graphicspath{ {Figures/} }

% Please include a clear, concise, and descriptive title
\title{ Do Motion Controls Effect Player Aggression?}

% Please do not change the subtitle
\subtitle{COMP230 - Ethics Essay}

% Please put your student number in the author field
\author{1507866}

\begin{document}
	
\maketitle
	
\abstract{This essay will look at motion controls in video games and how they effect the players aggression levels. It will specifically look at the Nintendo Wii and the Wiimote controller.}
	
\section{Introduction}
The effect of violent video games on player aggression is a commonly addressed subject and one that appears often in the media \cite{DailyMail, GuardianAggression, CBSNews}.
However, the evaluation of violent video games most commonly focuses on the acts of violence in the game and the graphics not the control schemes and mapping.  The intention of this essay is to look at whether the use of motion controls with natural mappings effects the player aggression levels.

\section{Violent Video Games and Player Aggression}
 
Modern video games have vastly improved graphics compared to games from previous generations  \cite{Fumhe}.  This has been accompanied by an increase in violent content in commercial video games \cite{Fumhe}. The improvement in graphics means that in game violence looks more realistic than in previous generations and is more immersive \cite{Kim}.

\bigskip
One issue could be with how violence in games is viewed and whether it is ethical. For example, an absolutist would say murder is wrong under any circumstance \cite{forsyth}.
Yet video game violence is a pretence, none of it is a real and the player is not harming a living being \cite{Tavinor}.
However, Forsyth's ideas about ethics date back to 1980 so may not be applicable to modern video games. 
This also raises the issue of whether video game violence should be viewed in the same way as real violence and therefore whether violence in video games is ethical. Motion controls complicate this further as the player is physically acting out the violence with the intention of harming a virtual character. 

\bigskip
There are many previous studies that have looked at the effects of video game violence and aggression. For example, Sherry found that video games do increase player's aggression levels. However, this rise is smaller than the effect of seeing violence on television  \cite{sherry2001effects}.
The rise in aggression is also short term from when the player starts playing a game. This implies that the rise in aggression fades. Even though video games can cause an increase in aggression this does not necessarily translate into aggressive behaviours \cite{Ferguson}. 

The Wii was released in 2006, the FBI crime statistics for America show an increase in violent crime that year \cite{FBI}.  However, there was a decrease when compared to the predicted trends for that year. The amount of violent crimes in America in 2006 was the third lowest in the last two decades \cite{FBIArchives}. This data suggests that the release of the Wii did not cause a significant rise in violence.


\section{Motion controls in video games}
Motion controls are control systems that monitor the player's movement. One type of motion control is kinesic natural mapping. This captures the player's movement without controllers or any physical props. Another type is incomplete tangible natural mapping. This version requires a physical object, such as the Wiimote. The Wiimote uses an accelerometer and infra-red to detect the player's movements. Both versions of motion control capture the player's movement and translate it into its in game equivalent \cite{McEwan2012}.

Both types of motion controls provide more intuitive and interactive controls \cite{Kim}.  However, they require the user to mimic real life action including violent ones. 

\bigskip
Improvements in technology means that video games continue to get more realistic, graphics and sound have been continually improving increasing the player's immersion in the game \cite{Kim}. Kim defines immersion as ``a psychological state or subjective perception in which even though part or all of an   individual’s   current   experience is generated by and/or filtered through human-made technology, part or all of the individual’s perception fails to accurately acknowledge the role of the technology in the experience" \cite{Kim}.

\bigskip
McArthur suggests that users may not want full immersion but instead ``neo-immersion" \cite{McArthur}. Whitson describes immersion in games as loss of  self, social and game awareness, in contrast Neo-Immersion focuses on those three factors \cite{Whitson}. Wii Sports is a game released with the Wii that uses movement controls using a Wiimote. Whitson suggests that this game is not trying to fully immerse the player they are likely to be playing it with friends and family. However, as this is only one game of many released on the Wii platform this idea cannot be applied to them all. This suggests that while motion controls make the game more interactive they also make it more social reducing conventional immersion.


Games such as Manhunt have been referred to as 'true murder simulators' when played on the Wii \cite{Manhunt}. This is due to the player having to mimic the violent acts performed in game.
The game has also been blamed for murder that was similar to violent acts in the game which led to the game being withdrawn from sale from UK stores \cite{Manhunt2Ban, ManhuntMurder}.


\section{Video Game Controls and Competence}
A possible cause for aggression when using motion controls could be how well an action or movement is translated from reality into a game.  

The Frustration --- Aggression hypothesis says that aggression is caused when a person is being blocked from reaching their goal \cite{dollard1939frustration}.  Przybylski \textit{et al} found links between lack of competence when playing a game and aggression. They say that competence-impeding controls ``have the potential to aggravate and demotivate players” \cite{przybylski}.

Games using motion controls are more intuitive as they use movement the player already knows \cite{Kim}. However, this could cause frustration for players who have experience in real life of what they are trying to accomplish in game. For example, if the player has experience with guns they may find a Wiimote does not work the same way making them feel incompetent. 
\bigskip
ADD TABLE OF WIIMOTE ACCURACY
\bigskip


McArthur \textit{et al}  found that the Wiimote accessory that held the closest resemblance to a gun had the highest error rate. However, there was another controller that was shaped similar to a gun that had the lowest error rate \cite{McArthur}.  However, their study did not include details to whether participants had previous experience with guns which could have effected the players accuracy. 

 
\bigskip

TV shows were once criticized for causing violence the same way video games now are \cite{sherry2001effects}.
%Motion controls were and evolution on from the standard controller.

Virtual Reality (VR) is a new emerging technology that attempts to fully immerse the player in a virtual world. VR consists of a Head Mounted Display (HMD) and often motion controllers similar to the Wiimote.
Looking at the previous evidence it would suggest that VR will soon be accused of causing aggression as well. Game developers Guerilla Games have said that their VR game R.I.G.S deliberately avoided violence and murder as they believe it to be too intense and the player cannot look away \cite{VRViolence}.
Polygon writer Kuchera says "Its' hard to imagine that people want to feel their virtual killing more acutely"     "The ability to hurt others isn't what drives my interest in virutal reality" \cite{PolygonVR} .

\section{Conclusion}
In conclusion the use of the Wiimote in violent games does not appear to cause more aggression. However, it could cause players to feel incompetent which leads to aggression. This incompetence could be due to poor mapping between the action and the movement required for the game or inaccurate controls. 

	
\bibliographystyle{ieeetr}
\bibliography{comp230_1}
	
\end{document}
