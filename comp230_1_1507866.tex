% Please do not change the document class
\documentclass{scrartcl}

% Please do not change these packages
\usepackage[hidelinks]{hyperref}
\usepackage[none]{hyphenat}
\usepackage{setspace}
\doublespace

% You may add additional packages here
\usepackage{amsmath}
\usepackage{graphicx} 
\graphicspath{ {Figures/} }

% Please include a clear, concise, and descriptive title
\title{The Effect of Motion Controls on Player Aggression }

% Please do not change the subtitle
\subtitle{COMP230 - Ethics Essay}

% Please put your student number in the author field
\author{1507866}

\begin{document}
	
\maketitle
	
\abstract{This essay will look at motion controls in video games and how they effect the players aggression levels. It will specifically look at the Nintendo Wii and the Wiimote controller.}
	
\section{Introduction}
The effect of violent video games on real life aggression is a commonly addressed subject and one that appears often in the media. However the violent video games looked at most commonly use a standard controller. The intention of this essay is to look a whether the use of motion controls in games effects the players aggression levels.


\section{Violent Video Games and Player Aggression}
 
Modern video games have vastly improved graphics compared to games from previous generations.  \cite{Fumhe}  This has been accompanied by an increase in the amount of violent content in commercial video games. \cite{Fumhe} The improvement in graphics means that in game violence looks more realistic than in previous generations and is likely more immersive. 

\bigskip
Different studies define violence in different ways. The issue is with how the in game violence is viewed and whether it is ethical. For example an absolutist would say murder is wrong under any circumstance. \cite{forsyth}
Yet video game violence is a pretence, none of it is a real and the player is not harming anyone. \cite{Tavinor}
This raises the issue of whether video game violence should be viewed in the same way as real violence and therefore whether violence in video games is ethical. Motion controls complicate this further as the player is physically acting out the violence with the intention of harming a virtual character. 

\bigskip
There are many previous studies that have looked at the effects of video game violence and aggression. For example Sherry found that video games do increase player's aggression levels. However this rise is smaller than the effect of seeing violence on television.  \cite{sherry2001effects}
The rise in aggression is also short term from when the player starts playing a game. This implies that the rise in aggression fades. Even though video games can cause an increase in aggression this does not necessarily translate into aggressive behaviours.  \cite{Ferguson}  


\section{Motion controls in video games}
Motion controls are control systems that monitor the players movement. One type of motion controls is kinesic natural mapping. This captures the players movement without controllers or any physical props. Another type is incomplete tangible natural mapping. This version requires a physical object, such as the Wiimote. Both versions capture that players movement and translate it into its virtual equivalent.  \cite{McEwan2012}

Both types of motion controls provide more intuitive and interactive controls. \cite{Kim}  However, they require the user to mimic real life action including violent ones. 


\bigskip
Improvements in technology means that video games continue to get more realistic, graphics and sound have been continually improving increasing the players immersion in the game \cite{Kim}. Kim defines immersion as ``a psychological state or subjective perception in which even though part or all of an   individual’s   current   experience is generated by and/or filtered through human-made technology, part or all of the individual’s perception fails to accurately acknowledge the role of the technology in the experience" \cite{Kim}.

\bigskip
McArthur suggests that users may not want full immersion but instead "neo-immersion". \cite{McArthur} Whitson describes immersion in games as loss of  self, social and game awarenesses, in contrast Neo-Immersion focuses on those three factors. \cite{Whitson} Wii Sports is a game released with the Wii that uses movement controls using a Wiimote, Whitson suggests that this game is not trying to fully immerse the player they are likely to be playing it with friends and family. However as this is only one game of many released on the Wii platform this idea can not be applied to them all. 
This suggests that while motion controls make the game more interactive it also make it more social...
\bigskip


\section{Video Game Controls and Competence}
A possible cause for aggression when using motion controls could be how well an action or movement is translated from reality into a game.  Przybylski \textit{et al} found links between lack of competence when playing a game and aggression. \cite{przybylski}

Games using motion controls are more intuitive as they use movement the player already knows. \cite{Kim} However this could cause frustration for players who have experience in real life of what they are trying to accomplish in game. For example if the player has experience with guns they may find a Wiimote does not work the same way making them feel incompetent. 

McArthur \textit{et al}  found that the Wiimote accessory that held the closest resemblance to a gun had the highest error rate. However, there was another controller that was shaped similar to a gun that had the lowest error rate. \cite{McArthur} .................
 
 
\bigskip
“Competence-impeding gaming experiences have the potential to aggravate and demotivate players”  Przybylski \textit{et al} \cite{przybylski}


The frustration --- aggression hypothesis says that aggression is caused when a person is being blocked from reaching their goal. \cite{dollard1939frustration}  


Realistic controllers \cite{McGloin}


 
\bigskip
TV shows were once criticized for for causing violence the same way video games now are. \cite{Sherry} 

Games such as Manhunt have been referred to as 'true murder simulators' when played on the Wii. \cite{Manhunt} This is due to the player having to mimic the violent acts performed in game.

Motion controls were and evolution on from the standard controller.

Virtual Reality (VR) is a new emerging technology that attempts to fully immerse the player in a virtual world. VR consists of a Head Mounted Display (HMD) and often motion controls similar to the Wiimote.

Looking at the previous evidence it would suggest that VR will soon be accused of causing aggression as well. 

\section{Conclusion}
In conclusion...

	
\bibliographystyle{ieeetr}
\bibliography{comp230_1}
	
\end{document}
